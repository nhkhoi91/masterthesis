% Chapter 1

\chapter{Introduction} % Main chapter title

\label{Chapter1} % For referencing the chapter elsewhere, use \ref{Chapter1} 

%----------------------------------------------------------------------------------------

% Define some commands to keep the formatting separated from the content 
\newcommand{\keyword}[1]{\textbf{#1}}
\newcommand{\tabhead}[1]{\textbf{#1}}
\newcommand{\code}[1]{\texttt{#1}}
\newcommand{\file}[1]{\texttt{\bfseries#1}}
\newcommand{\option}[1]{\texttt{\itshape#1}}

%----------------------------------------------------------------------------------------

\section{A statement of the problem}
Recommender system has been an interesting subject of research for long. The idea of recommender system began in early 1990s - with the popularization of the Internet, to utilize the critique of millions of people online to help us acquire more useful and interesting content. The PARC Tapestry system \cite{goldberg1992using} first introduced the novel idea of collaborative filtering technique in 1992, which instantaneously became a subject of interest for many other research groups and was employed in many systems, including the GroupLens system \cite{resnick1994open}, the Ringo system at MIT \cite{shardanand1995social}, and the Bellcore Video Recommender \cite{hill1995recommending}. As the domain gained more attractions, various approaches and techniques have been developed in the field, such as the materialization of content-based approach in 1997 \cite{balabanovic1997content}, or the knowledge-based approach that is embedded in the FindMe recommender system \cite{trewin2000knowledge}. 

Music recommender system is a branch, among others, of the recommender domain. The purpose of a music recommender is to suggest songs that user might like amid a collection of millions of sound tracks. Many approaches have been proposed, including collaborative filtering to exploit the wisdom of the crowd, content-based approach to find similarity tracks, genre, or artist, and context-based approach to explore the relationship between context and listeners' behaviors. 

Often, for content-based approach, acoustic features such as timbral are applied for genre classification. However, this approach looks at music recommendation as a purely signal processing problem, without taking into consider the intrinsic properties that humans perceive. Because acoustic features cannot capture semantic meaning of a track, an "energetic" song can be played next to a "nostalgic" one because of the similarity of the instruments, leading to a poor recommendation.

Semantic annotation is another promising trend in content-based approach. The idea of these approaches is to bridge the semantic gaps between acoustic features using machine learning techniques. This is, however, a complex problem, as mapping between human annotations and acoustic features is not easy to define \cite{aucouturier2009sounds}. Many attempts have been made, such as the one that trains Gaussian mixture models of MFCCs for finding genre, moods, and instruments. 

The Echo Nest \footnote{http://the.echonest.com/}, a music intelligence and data platform company, took another approach. Instead of exploring genres and moods similarity, the company extracts other kind of music features. Besides some well-known ones such as tempo and key, it also defines some more specialize ones, such as speechiness to denote the degree of spoken words, or danceability to describe how suitable a track is for dancing. 

As Spotify \footnote{https://www.spotify.com}, a digital music service, acquired The Echo Nest \cite{press_2014}, the company attached Echo Nest's features into its own music library, creating a database of audio features for millions of track. As Spotify published the database \footnote{https://developer.spotify.com/web-api/get-audio-features/}, it creates chances for researchers and developers to develop an insight on the correlation between these variables and users' tastes, as well as to have a possibility to increase the quality of music recommendations.

However, Spotify does not provide user listening data. To overcome this difficulty, I tried to combine the famous last.fm \footnote{https://www.last.fm/} dataset with the one from Spotify. As I decide to work with album as the unit item, user data from last.fm is crawled on the album level. Of an album, all the tracks that exist as well as their acoustic features are then acquired from Spotify. The mean of the features of the tracks of an album is taken as the representations of that album. More information on the construction of the dataset will be described in Chapter 4. 

In this thesis, the following tasks are performed: 

\begin{itemize}
   \item[•] A join dataset, taking user listening history from Last.fm and track acoustic features from Spotify, is constructed. 
	\item[•] Explanatory data analysis, including the use of descriptive statistics and clustering analysis, is conducted to understand the property and characteristics of the dataset.
	\item[•] Three machine learning algorithms, including a pure collaborative filtering, a state of the art collaborative filtering, and a hybrid method using learning to rank algorithm, are applied on the dataset. The purpose is to observe the effect of the acoustic features on the recommendation of the top 5 items, compare to the state of the art collaborative filtering and the na\"ive collaborative filtering.
\end{itemize}


The following of the thesis is structured as follow: Chapter 2 describes fundamental approaches in recommender system and provides an overview of current music recommender's research. Chapter 3 describes the problem and the construction of the dataset. Chapter 4 describes in detail the algorithms, while Chapter 5 explains the experiments and details the result. Chapter 6 summarizes the thesis, as well as stating future works.


%----------------------------------------------------------------------------------------


